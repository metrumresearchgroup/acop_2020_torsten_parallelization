%----------------------------------------------------------------------------------------
%	PACKAGES AND OTHER DOCUMENT CONFIGURATIONS
%----------------------------------------------------------------------------------------

\documentclass[landscape,a0paper,fontscale=0.45]{baposter} % Adjust the font scale/size here

\usepackage{graphicx} % Required for including images
\graphicspath{{figures/}} % Directory in which figures are stored

\usepackage{amsmath} % For typesetting math
\usepackage{amssymb} % Adds new symbols to be used in math mode

\usepackage{booktabs} % Top and bottom rules for tables
\usepackage{enumitem} % Used to reduce itemize/enumerate spacing
\usepackage{palatino} % Use the Palatino font
\usepackage[font=small,labelfont=bf]{caption} % Required for specifying captions to tables and figures

\usepackage[hyphens]{url}
\usepackage{multicol} % Required for multiple columns
\setlength{\columnsep}{1.5em} % Slightly increase the space between columns
\setlength{\columnseprule}{0mm} % No horizontal rule between columns

\usepackage{tikz} % Required for flow chart
\usetikzlibrary{shapes,arrows} % Tikz libraries required for the flow chart in the template

\newcommand{\compresslist}{ % Define a command to reduce spacing within itemize/enumerate environments, this is used right after \begin{itemize} or \begin{enumerate}
\setlength{\itemsep}{1pt}
\setlength{\parskip}{0pt}
\setlength{\parsep}{0pt}
}

\definecolor{MRGGreen}{rgb}{0, 0.350, 0.200}

\newenvironment{ColFigure}
  {\par\medskip\noindent\minipage{\linewidth}}
  {\endminipage\par\medskip}


\begin{document}

\begin{poster}
{
headerborder=closed, % Adds a border around the header of content boxes
colspacing=1em, % Column spacing
borderColor=MRGGreen, % Border color
headerColorOne=MRGGreen, % Background color for the header in the content boxes (left side)
headerColorTwo=MRGGreen, % Background color for the header in the content boxes (right side)
headerFontColor=MRGGreen, % Text color for the header text in the content boxes
boxColorOne=white, % Background color of the content boxes
textborder=roundedleft, % Format of the border around content boxes, can be: none, bars, coils, triangles, rectangle, rounded, roundedsmall, roundedright or faded
eyecatcher=true, % Set to false for ignoring the left logo in the title and move the title left
headerheight=0.1\textheight, % Height of the header
headershape=roundedright, % Specify the rounded corner in the content box headers, can be: rectangle, small-rounded, roundedright, roundedleft or rounded
headerfont=\Large\bf\textsc, % Large, bold and sans serif font in the headers of content boxes
%textfont={\setlength{\parindent}{1.5em}}, % Uncomment for paragraph indentation
linewidth=2pt % Width of the border lines around content boxes
}
%----------------------------------------------------------------------------------------
%	TITLE SECTION 
%----------------------------------------------------------------------------------------
%
{\includegraphics[height=4em]{graphics/logo.jpg}} % First university/lab logo on the left
{\bf{Speed up populational Bayesian inference by combining cross-chain warmup and within-chain parallelization}\vspace{0.8em}} % Poster title
{\textsc{ Yi Zhang\textsuperscript{1}, William R. Gillespie\textsuperscript{1}, Ben Bales\textsuperscript{2}, Aki Vehtari\textsuperscript{3} \hspace{12pt} \normalsize{1. Metrum Research Group, 2. Columbia University, 3. Aalto University}}} % Author names and institution
{\includegraphics[height=6em]{graphics/alumnae_columbia.png}\\
\includegraphics[height=4em]{graphics/Aalto_EN_13_BLACK_1_Original.png}
} % Second university/lab logo on the right


%----------------------------------------------------------------------------------------
%	Objectives
%----------------------------------------------------------------------------------------

\headerbox{Objectives}{name=objective,column=0,row=0,span=2}{
% \begin{multicols}{2}
% \vspace{1em}
\begin{itemize}
\item Improve the performance of Bayesian inference of population
  models by designing a multilevel parallel scheme that combines
  a cross-chain warmup algorithm in probabilistic programming language
  Stan \cite{carpenter_stan_2017} and Torsten's within-chain parallel
  group ODE solvers \cite{Torsten}\cite{torsten_pmx_group}.
\item Demonstrate that the new approach significantly improves large PKPD model simulation efficiency.
\end{itemize}

}

%----------------------------------------------------------------------------------------
%	Cross-chain
%----------------------------------------------------------------------------------------

\headerbox{Cross-chain warmup}{name=warmup,column=0,span=2,below=objective,above=bottom,aligned=conclusion}{
\begin{multicols}{2}
\vspace{1em}
The efficacy of standard Stan practice of running a fixed number of warmup
iterations is unknown \emph{a priori}. We propose a new algorithm
in order to evaluate warmup quality dynamically and avoid unncessary iterations, by checking potential scale reduction
coefficients (\(\hat{R}\)) and effective sample sizes (ESS)
\cite{vehtari_rank-normalization_2019}:
\begin{ColFigure}
 \centering
 \includegraphics[width=\linewidth]{figure/cross_chain_diagram.pdf}
 \captionof{figure}{\emph{Cross-chain} algorithm is named after the fact that the chains communicate regularly during warmup.}
\end{ColFigure}

\begin{enumerate}
\item With a fixed window size \(w\), initiate warmup with stepsize adaptation.
\item At the end of a window, aggregate joint posterior probability from all the chains and calculate corresponding \(\hat{R}\) and ESS. 
For example, with default window size \(w=100\), when warmup reaches iteration 200, calculate
\(\hat{R}^i\) and \(\text{ESS}^i\) for \(i=1, 2\), so that
\(\hat{R}^1\) and \(\text{ESS}^1\) are based on warmup iteration 1 to
200, and \(\hat{R}^2\) and \(\text{ESS}^2\) are based on warmup iteration 101 to 200.
\item At the end of window \(n\), with predefined target value
  \(\hat{R}^{0}\) and ESS\(^{0}\), from \({1, \dots, n}\), select \(j\) with maximum $\text{ESS}^j$,
and calculate a new metric using samples from corresponding
windows. Determine \emph{convergence} by checking if $\hat{R}^j < \hat{R}^0$ and $\text{ESS}^j > \text{ESS}^0$. 
If converges, move to post-warmup sampling, otherwise repeat step 2.
\end{enumerate}


\vspace{1em}
Benchmarks are performed with different target ESS and
regular Stan run(1000 warmup iterations).
We run each setup with 10 PRNG seeds and collect average(barplot) and
standard deviation(error bar) of the following quantities.
\begin{center}
\footnotesize
\begin{tabular}{l}
\hline
total number of leapfrog integration steps in warmup \\
total number of leapfrog integration steps in sampling \\
number of leapfrog integration steps in per each warmup iteration \\
number of leapfrog integration steps in per each sampling iteration \\
minimum ESS\(_{\text{bulk}}\) per iteration \\
minimum ESS\(_{\text{tail}}\) per iteration \\
minimum ESS\(_{\text{bulk}}\) per leapfrog step \\
minimum ESS\(_{\text{tail}}\) per leapfrog step \\
maximum wall time(in seconds) \\
\hline
\end{tabular}
\end{center}

\begin{ColFigure}
\centering
\includegraphics[width=0.95\linewidth]{./figure/cross_chain_ess_effect_arK.pdf}
\captionof{figure}{Cross-chain warmup performance: arK model\cite{posteriordb}}
\end{ColFigure}

\begin{ColFigure}
\centering
\includegraphics[width=0.95\linewidth]{./figure/cross_chain_ess_effect_eight_schools.pdf}
\captionof{figure}{Cross-chain warmup performance: eight schools model\cite{posteriordb}}
\end{ColFigure}

\begin{ColFigure}
\centering
\includegraphics[width=0.95\linewidth]{./figure/cross_chain_ess_effect_sblrc-blr.pdf}
\captionof{figure}{Cross-chain warmup performance: sblrc-blr model\cite{posteriordb}}
\end{ColFigure}

\begin{ColFigure}
\centering
\includegraphics[width=0.95\linewidth]{./figure/cross_chain_ess_effect_sir.pdf}
\captionof{figure}{Cross-chain warmup performance: SIR model\cite{posteriordb}}
\end{ColFigure}


% \begin{ColFigure}
% \centering
% \includegraphics[width=0.7\linewidth]{./figure/cross_chain_ess_effect_arK.png}
% \caption{Cross-chain warmup performance comparison: arK model}
% \end{ColFigure}


% \begin{ColFigure}
% \centering
% \includegraphics[width=0.7\linewidth]{./figure/cross_chain_ess_effect_chem.png}
% \caption{Cross-chain warmup performance comparison: chemical reaction model}
% \end{ColFigure}

\end{multicols}
}

%----------------------------------------------------------------------------------------
%	multilevel: method
%----------------------------------------------------------------------------------------

\headerbox{Multilevel parallelization: cross-chain warmup + within-chain parallelization}{name=multilevel,column=2,span=2,row=0}{

\begin{multicols}{2}
\vspace{1em}
\subsection*{Method}
\begin{center}
\includegraphics[width=\linewidth]{./figure/within_chain_parallel_diagram.pdf}
\captionof{figure}{Multilevel parallelism for ODE-based population models. A simplified version of Figure 1, the lower diagram shows the cross-chain warmup through multiple windows. In within-chain parallelization, as shown in the upper diagram, each chain has its own parameter samples(indicated by different colors), and dedicated processes for solving the population model. Thus the parallel computing is based on cross-chain level and within-chain level.\label{multilevel-diagram}}
\end{center}

\begin{center}
\begin{tabular}{l l l}
\hline
Level & Parallel operation & Parallel communication \\
\hline
1 & Cross-chain warmup & At the end of warmup windows \\
2 & Within-chain parallel ODE solver & During likelihhood evaluation \\
\hline
\end{tabular}
\captionof{table}{A framework of
\emph{multilevel parallelism} for Bayesian inference of population models. \label{tab:multilevel}}
\end{center}

\subsection*{Example}
We consider a time-to-event model for the time to the first grade 2+ peripheral neuropathy (PN)
event in patients treated with an antibody-drug conjugate (ADC)
delivering monomethyl auristatin E (MMAE). We call it
Time-To-PN(TTPN) model, and analyze data using a
simplified version of the model reported in
\cite{lu_time--event_2017}. We consider three treatment arms:
fauxlatuzumab vedotin 1.2, 1.8 and 2.4 mg/kg IV boluses q3w x 6 doses,
with 20 patients per treatment arm. In this model,
each patient's PK is described by an effective compartment model(one-compartment),
and PD by a linear model. The likelihood for time to first 2+ PN event
is described by a hazard function that depends on the concentration
effect through Weibull distribution. Two unknowns from
PK model and the cumulative hazard form a three-component
ODE system. Each evaluation of likelihood requires solving this
3-system for every patient. 

ODEs corresponding to the entire
population are solved by a single call of Torsten function \texttt{\texttt{pmx\_solve\_group\_rk45}}. The three parameters of the
Torstenn model are:
\begin{itemize}
\item \(k_{e0}\) in effective compartment model.
\item \(\alpha\) the coefficient of linear PD model.
\item \(\beta\) Weibull distribution scale parameter.
\end{itemize}

\subsubsection*{Warmup quality}
Table \ref{tab:ttpn} shows cross-chain and
regular run performance(target ESS = 400). Consistent with the
previous benchmark models, the cross-chain warmup reduce total run time without
compromising ESS, leading to ~15\% wall time improvement.

\begin{center}
\footnotesize
\begin{tabular}{l l l}
\hline
 & Cross-chain & Regular \\
\hline
 \texttt{leapfrogs(warmup)}        &  \texttt{1.002e+04}  & \texttt{1.588e+04}\\
 \texttt{leapfrogs(sampling)}      &  \texttt{1.709e+04}  & \texttt{1.831e+04}\\
 \texttt{leapfrogs(warmup)/iter}   &  \texttt{1.822e+01}  & \texttt{1.588e+01}\\
 \texttt{leapfrogs(sampling)/iter} &  \texttt{1.709e+01}  & \texttt{1.831e+01}\\
 \texttt{min(bulk\_ESS/iter)}       & \texttt{2.805e-01}  & \texttt{2.340e-01}\\
 \texttt{min(tail\_ESS/iter)}       & \texttt{3.482e-01}  & \texttt{3.205e-01}\\
 \texttt{min(bulk\_ESS/leapfrog)}   & \texttt{1.641e-02}  & \texttt{1.277e-02}\\
 \texttt{min(tail\_ESS/leapfrog)}   & \texttt{2.037e-02}  & \texttt{1.749e-02}\\
 \texttt{max(elapsed\_time)}        & \texttt{1.702e+03}  & \texttt{1.979e+03}\\
\hline
\end{tabular}
\captionof{table}{Cross-chain runs vs regular runs(target ESS=400)}
\label{tab:ttpn}
\end{center}

\subsubsection*{Parallel speedup}
Speedup is investigated by running the model
with 4 chains using \(n_{\text{proc}} = 4, 8, 16, 32, 60\)
processes. Equivalently, there are
1, 2, 4, 8, 15 processes per chain for within-chain parallelization.
With population size 60, this is also equivalent to have each process handle the solution of
60, 30, 15, 8, 4 subjects' ODE system, respectively.
\begin{itemize}
\item Both muiltilevel and within-chain-only parallel runs scale near-linearly up to 60
  processes(15 processes per chain \(\times\) 4 chains).
\item In the range of \(n_{\text{proc}}=32, 60, 80\), multilevel runs
  exhibit a steady \textasciitilde 20\% performance improvement, completely
  contributed by cross-chain warmup.
\end{itemize}

\begin{center}
\includegraphics[width=\linewidth]{./figure/ttpn2_perf_benchmark.pdf}
\captionof{figure}{Multilevel scheme parallel performance: TTPN
  model(target ESS=400). Wall time speedup uses regular Stan run as reference. With all
runs having 1000 post-warmup sampling iterations, in
multilevel runs the number of warmup iterations is determinted at
runtime, while both within-chain parallel runs and regular Stan runs
have 1000 warmup iterations. Among 4 chains in a run, we use the
one with maximum total walltime(in seconds) as performance measure, as
in practice usually further model evaluation becomes accessible only
after all chains finish.}
\label{fig:ttpn_speedup}
\end{center}

\end{multicols}
\vspace{0.5em}
}

%----------------------------------------------------------------------------------------
%	REFERENCES
%----------------------------------------------------------------------------------------

\headerbox{References}{name=references,column=2,below=multilevel,above=bottom}{
\renewcommand{\section}[2]{\vskip 0.05em} % Get rid of the default "References" section title
\footnotesize{ % Reduce the font size in this block
\bibliographystyle{siam}
\bibliography{torsten} % Use torsten.bib as the bibliography file
}}

%----------------------------------------------------------------------------------------
%	CONCLUSION
%----------------------------------------------------------------------------------------

\headerbox{Conclusion and future work}{name=conclusion,column=3,below=multilevel}{
  \begin{itemize}
  \item Multilevel parallelism significantly improves computational efficiency and extends
    the range of models that may be practically implemented.
  \item Our planed follow-up study is to seek higher efficiency by maintaining
    target ESS while increasing the number of parallel chains during warmup.
  \end{itemize}
}

\headerbox{See also}{name=github_repo,headerborder=none,textborder=none,column=3,below=conclusion,above=bottom} {
\begin{multicols}{2}
\url{https://github.com/metrumresearchgroup/acop_2020_torsten_parallelization}
for more benchmark examples and source code.
\begin{flushright}
\includegraphics[width=0.4\linewidth]{./figure/github_repo_qr_code.png}
\end{flushright}
\end{multicols}  
}

%----------------------------------------------------------------------------------------
%	CONTACT INFORMATION
%----------------------------------------------------------------------------------------

% \headerbox{Contact Information}{name=contact,column=3,aligned=references,above=bottom}{ % This block is as tall as the references block

% \begin{description}\compresslist
% \item[Web] www.university.edu/smithlab
% \item[Email] john@smith.com
% \item[Phone] +1 (000) 111 1111
% \end{description}
% }




%----------------------------------------------------------------------------------------

\end{poster}

\end{document}
